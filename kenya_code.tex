% Options for packages loaded elsewhere
\PassOptionsToPackage{unicode}{hyperref}
\PassOptionsToPackage{hyphens}{url}
%
\documentclass[
]{article}
\usepackage{amsmath,amssymb}
\usepackage{iftex}
\ifPDFTeX
  \usepackage[T1]{fontenc}
  \usepackage[utf8]{inputenc}
  \usepackage{textcomp} % provide euro and other symbols
\else % if luatex or xetex
  \usepackage{unicode-math} % this also loads fontspec
  \defaultfontfeatures{Scale=MatchLowercase}
  \defaultfontfeatures[\rmfamily]{Ligatures=TeX,Scale=1}
\fi
\usepackage{lmodern}
\ifPDFTeX\else
  % xetex/luatex font selection
\fi
% Use upquote if available, for straight quotes in verbatim environments
\IfFileExists{upquote.sty}{\usepackage{upquote}}{}
\IfFileExists{microtype.sty}{% use microtype if available
  \usepackage[]{microtype}
  \UseMicrotypeSet[protrusion]{basicmath} % disable protrusion for tt fonts
}{}
\makeatletter
\@ifundefined{KOMAClassName}{% if non-KOMA class
  \IfFileExists{parskip.sty}{%
    \usepackage{parskip}
  }{% else
    \setlength{\parindent}{0pt}
    \setlength{\parskip}{6pt plus 2pt minus 1pt}}
}{% if KOMA class
  \KOMAoptions{parskip=half}}
\makeatother
\usepackage{xcolor}
\usepackage[margin=1in]{geometry}
\usepackage{color}
\usepackage{fancyvrb}
\newcommand{\VerbBar}{|}
\newcommand{\VERB}{\Verb[commandchars=\\\{\}]}
\DefineVerbatimEnvironment{Highlighting}{Verbatim}{commandchars=\\\{\}}
% Add ',fontsize=\small' for more characters per line
\usepackage{framed}
\definecolor{shadecolor}{RGB}{248,248,248}
\newenvironment{Shaded}{\begin{snugshade}}{\end{snugshade}}
\newcommand{\AlertTok}[1]{\textcolor[rgb]{0.94,0.16,0.16}{#1}}
\newcommand{\AnnotationTok}[1]{\textcolor[rgb]{0.56,0.35,0.01}{\textbf{\textit{#1}}}}
\newcommand{\AttributeTok}[1]{\textcolor[rgb]{0.13,0.29,0.53}{#1}}
\newcommand{\BaseNTok}[1]{\textcolor[rgb]{0.00,0.00,0.81}{#1}}
\newcommand{\BuiltInTok}[1]{#1}
\newcommand{\CharTok}[1]{\textcolor[rgb]{0.31,0.60,0.02}{#1}}
\newcommand{\CommentTok}[1]{\textcolor[rgb]{0.56,0.35,0.01}{\textit{#1}}}
\newcommand{\CommentVarTok}[1]{\textcolor[rgb]{0.56,0.35,0.01}{\textbf{\textit{#1}}}}
\newcommand{\ConstantTok}[1]{\textcolor[rgb]{0.56,0.35,0.01}{#1}}
\newcommand{\ControlFlowTok}[1]{\textcolor[rgb]{0.13,0.29,0.53}{\textbf{#1}}}
\newcommand{\DataTypeTok}[1]{\textcolor[rgb]{0.13,0.29,0.53}{#1}}
\newcommand{\DecValTok}[1]{\textcolor[rgb]{0.00,0.00,0.81}{#1}}
\newcommand{\DocumentationTok}[1]{\textcolor[rgb]{0.56,0.35,0.01}{\textbf{\textit{#1}}}}
\newcommand{\ErrorTok}[1]{\textcolor[rgb]{0.64,0.00,0.00}{\textbf{#1}}}
\newcommand{\ExtensionTok}[1]{#1}
\newcommand{\FloatTok}[1]{\textcolor[rgb]{0.00,0.00,0.81}{#1}}
\newcommand{\FunctionTok}[1]{\textcolor[rgb]{0.13,0.29,0.53}{\textbf{#1}}}
\newcommand{\ImportTok}[1]{#1}
\newcommand{\InformationTok}[1]{\textcolor[rgb]{0.56,0.35,0.01}{\textbf{\textit{#1}}}}
\newcommand{\KeywordTok}[1]{\textcolor[rgb]{0.13,0.29,0.53}{\textbf{#1}}}
\newcommand{\NormalTok}[1]{#1}
\newcommand{\OperatorTok}[1]{\textcolor[rgb]{0.81,0.36,0.00}{\textbf{#1}}}
\newcommand{\OtherTok}[1]{\textcolor[rgb]{0.56,0.35,0.01}{#1}}
\newcommand{\PreprocessorTok}[1]{\textcolor[rgb]{0.56,0.35,0.01}{\textit{#1}}}
\newcommand{\RegionMarkerTok}[1]{#1}
\newcommand{\SpecialCharTok}[1]{\textcolor[rgb]{0.81,0.36,0.00}{\textbf{#1}}}
\newcommand{\SpecialStringTok}[1]{\textcolor[rgb]{0.31,0.60,0.02}{#1}}
\newcommand{\StringTok}[1]{\textcolor[rgb]{0.31,0.60,0.02}{#1}}
\newcommand{\VariableTok}[1]{\textcolor[rgb]{0.00,0.00,0.00}{#1}}
\newcommand{\VerbatimStringTok}[1]{\textcolor[rgb]{0.31,0.60,0.02}{#1}}
\newcommand{\WarningTok}[1]{\textcolor[rgb]{0.56,0.35,0.01}{\textbf{\textit{#1}}}}
\usepackage{graphicx}
\makeatletter
\def\maxwidth{\ifdim\Gin@nat@width>\linewidth\linewidth\else\Gin@nat@width\fi}
\def\maxheight{\ifdim\Gin@nat@height>\textheight\textheight\else\Gin@nat@height\fi}
\makeatother
% Scale images if necessary, so that they will not overflow the page
% margins by default, and it is still possible to overwrite the defaults
% using explicit options in \includegraphics[width, height, ...]{}
\setkeys{Gin}{width=\maxwidth,height=\maxheight,keepaspectratio}
% Set default figure placement to htbp
\makeatletter
\def\fps@figure{htbp}
\makeatother
\setlength{\emergencystretch}{3em} % prevent overfull lines
\providecommand{\tightlist}{%
  \setlength{\itemsep}{0pt}\setlength{\parskip}{0pt}}
\setcounter{secnumdepth}{-\maxdimen} % remove section numbering
\usepackage{booktabs}
\usepackage{caption}
\usepackage{longtable}
\usepackage{colortbl}
\usepackage{array}
\usepackage{multirow}
\usepackage{multicol}
\usepackage{hhline}
\newlength\Oldarrayrulewidth
\newlength\Oldtabcolsep
\usepackage{hyperref}
\usepackage{float}
\usepackage{wrapfig}
\ifLuaTeX
  \usepackage{selnolig}  % disable illegal ligatures
\fi
\IfFileExists{bookmark.sty}{\usepackage{bookmark}}{\usepackage{hyperref}}
\IfFileExists{xurl.sty}{\usepackage{xurl}}{} % add URL line breaks if available
\urlstyle{same}
\hypersetup{
  pdftitle={kenya\_code},
  pdfauthor={Dianovics Dominik},
  hidelinks,
  pdfcreator={LaTeX via pandoc}}

\title{kenya\_code}
\author{Dianovics Dominik}
\date{2023-12-15}

\begin{document}
\maketitle

\begin{verbatim}
## Warning: package 'tidytuesdayR' was built under R version 4.3.2
\end{verbatim}

\begin{verbatim}
## -- Attaching core tidyverse packages ------------------------ tidyverse 2.0.0 --
## v dplyr     1.1.2     v readr     2.1.4
## v forcats   1.0.0     v stringr   1.5.0
## v ggplot2   3.4.2     v tibble    3.2.1
## v lubridate 1.9.2     v tidyr     1.3.0
## v purrr     1.0.1     
## -- Conflicts ------------------------------------------ tidyverse_conflicts() --
## x dplyr::filter() masks stats::filter()
## x dplyr::lag()    masks stats::lag()
## i Use the conflicted package (<http://conflicted.r-lib.org/>) to force all conflicts to become errors
\end{verbatim}

\begin{verbatim}
## Warning: package 'stringdist' was built under R version 4.3.2
\end{verbatim}

\begin{verbatim}
## 
## Attaching package: 'stringdist'
## 
## The following object is masked from 'package:tidyr':
## 
##     extract
\end{verbatim}

\begin{verbatim}
## Warning: package 'fuzzyjoin' was built under R version 4.3.2
\end{verbatim}

\begin{verbatim}
## Warning: package 'patchwork' was built under R version 4.3.1
\end{verbatim}

\begin{verbatim}
## Warning: package 'gt' was built under R version 4.3.2
\end{verbatim}

\begin{verbatim}
## 
## Attaching package: 'gridExtra'
## 
## The following object is masked from 'package:dplyr':
## 
##     combine
## 
## Loading required package: zoo
## 
## Attaching package: 'zoo'
## 
## The following objects are masked from 'package:base':
## 
##     as.Date, as.Date.numeric
\end{verbatim}

\begin{verbatim}
## Warning: package 'robustbase' was built under R version 4.3.2
\end{verbatim}

\begin{verbatim}
## 
## Attaching package: 'MASS'
## 
## The following object is masked from 'package:patchwork':
## 
##     area
## 
## The following object is masked from 'package:dplyr':
## 
##     select
\end{verbatim}

\begin{verbatim}
## Warning: package 'mediation' was built under R version 4.3.2
\end{verbatim}

\begin{verbatim}
## Loading required package: Matrix
## 
## Attaching package: 'Matrix'
## 
## The following objects are masked from 'package:tidyr':
## 
##     expand, pack, unpack
## 
## Loading required package: mvtnorm
## Loading required package: sandwich
## mediation: Causal Mediation Analysis
## Version: 4.5.0
\end{verbatim}

\begin{verbatim}
## Warning: package 'devtools' was built under R version 4.3.2
\end{verbatim}

\begin{verbatim}
## Loading required package: usethis
\end{verbatim}

\begin{verbatim}
## Warning: package 'usethis' was built under R version 4.3.2
\end{verbatim}

\begin{verbatim}
## 
## Attaching package: 'flexplot'
## 
## The following object is masked from 'package:ggplot2':
## 
##     flip_data
\end{verbatim}

\begin{Shaded}
\begin{Highlighting}[]
\NormalTok{data }\OtherTok{\textless{}{-}} \FunctionTok{tt\_load}\NormalTok{(}\StringTok{\textquotesingle{}2021{-}01{-}19\textquotesingle{}}\NormalTok{)}
\end{Highlighting}
\end{Shaded}

\begin{verbatim}
## 
##  Downloading file 1 of 3: `households.csv`
##  Downloading file 2 of 3: `crops.csv`
##  Downloading file 3 of 3: `gender.csv`
\end{verbatim}

\begin{Shaded}
\begin{Highlighting}[]
\NormalTok{gender }\OtherTok{\textless{}{-}}\NormalTok{ data}\SpecialCharTok{$}\NormalTok{gender}
\NormalTok{crops }\OtherTok{\textless{}{-}}\NormalTok{ data}\SpecialCharTok{$}\NormalTok{crops}
\NormalTok{households }\OtherTok{\textless{}{-}}\NormalTok{ data}\SpecialCharTok{$}\NormalTok{households}
\end{Highlighting}
\end{Shaded}

\begin{Shaded}
\begin{Highlighting}[]
\FunctionTok{view}\NormalTok{(crops)}
\FunctionTok{str}\NormalTok{(crops)}
\end{Highlighting}
\end{Shaded}

\begin{verbatim}
## spc_tbl_ [48 x 11] (S3: spec_tbl_df/tbl_df/tbl/data.frame)
##  $ SubCounty   : chr [1:48] "KENYA" "MOMBASA" "KWALE" "KILIFI" ...
##  $ Farming     : num [1:48] 6354211 12497 108074 161188 35094 ...
##  $ Tea         : num [1:48] 476613 NA NA NA NA ...
##  $ Coffee      : num [1:48] 478936 NA NA NA NA ...
##  $ Avocado     : num [1:48] 966976 NA 1063 NA NA ...
##  $ Citrus      : num [1:48] 177445 NA 10053 6808 1109 ...
##  $ Mango       : num [1:48] 796867 NA 30272 37519 6561 ...
##  $ Coconut     : num [1:48] 90952 1688 31954 47561 2228 ...
##  $ Macadamia   : num [1:48] 195999 NA 881 NA NA ...
##  $ Cashew Nut  : num [1:48] 61664 602 22803 27940 1691 ...
##  $ Khat (Miraa): num [1:48] 134148 NA NA NA NA ...
##  - attr(*, "spec")=
##   .. cols(
##   ..   SubCounty = col_character(),
##   ..   Farming = col_double(),
##   ..   Tea = col_double(),
##   ..   Coffee = col_double(),
##   ..   Avocado = col_double(),
##   ..   Citrus = col_double(),
##   ..   Mango = col_double(),
##   ..   Coconut = col_double(),
##   ..   Macadamia = col_double(),
##   ..   `Cashew Nut` = col_double(),
##   ..   `Khat (Miraa)` = col_double()
##   .. )
##  - attr(*, "problems")=<externalptr>
\end{verbatim}

\begin{Shaded}
\begin{Highlighting}[]
\CommentTok{\#Issue with crops, all names are caps lock}

\NormalTok{crops }\OtherTok{\textless{}{-}}\NormalTok{ crops }\SpecialCharTok{|\textgreater{}} 
  \FunctionTok{mutate}\NormalTok{(}
    \AttributeTok{SubCounty =} \FunctionTok{str\_to\_lower}\NormalTok{(SubCounty),}
    \AttributeTok{SubCounty =} \FunctionTok{str\_to\_title}\NormalTok{(SubCounty)}
\NormalTok{  )}

\CommentTok{\#Removed the Khat crop because I don\textquotesingle{}t know what that is and there are too many missing values}

\NormalTok{crops }\OtherTok{\textless{}{-}}\NormalTok{ crops[, }\SpecialCharTok{!}\FunctionTok{colnames}\NormalTok{(crops) }\SpecialCharTok{\%in\%} \StringTok{"Khat (Miraa)"}\NormalTok{]}


\FunctionTok{view}\NormalTok{(households)}
\FunctionTok{str}\NormalTok{(households)}
\end{Highlighting}
\end{Shaded}

\begin{verbatim}
## spc_tbl_ [48 x 4] (S3: spec_tbl_df/tbl_df/tbl/data.frame)
##  $ County              : chr [1:48] "Kenya   " "Mombasa   " "Kwale  " "Kilifi  " ...
##  $ Population          : num [1:48] 47213282 1190987 858748 1440958 314710 ...
##  $ NumberOfHouseholds  : num [1:48] 12143913 378422 173176 298472 68242 ...
##  $ AverageHouseholdSize: num [1:48] 3.9 3.1 5 4.8 4.6 3.7 3.5 5.9 6.1 6.9 ...
##  - attr(*, "spec")=
##   .. cols(
##   ..   County = col_character(),
##   ..   Population = col_double(),
##   ..   NumberOfHouseholds = col_double(),
##   ..   AverageHouseholdSize = col_double()
##   .. )
##  - attr(*, "problems")=<externalptr>
\end{verbatim}

\begin{Shaded}
\begin{Highlighting}[]
\FunctionTok{view}\NormalTok{(gender)}
\FunctionTok{str}\NormalTok{(gender)}
\end{Highlighting}
\end{Shaded}

\begin{verbatim}
## spc_tbl_ [48 x 5] (S3: spec_tbl_df/tbl_df/tbl/data.frame)
##  $ County  : chr [1:48] "Total" "Mombasa" "Kwale" "Kilifi" ...
##  $ Male    : num [1:48] 23548056 610257 425121 704089 158550 ...
##  $ Female  : num [1:48] 24014716 598046 441681 749673 157391 ...
##  $ Intersex: num [1:48] 1524 30 18 25 2 ...
##  $ Total   : num [1:48] 47564296 1208333 866820 1453787 315943 ...
##  - attr(*, "spec")=
##   .. cols(
##   ..   County = col_character(),
##   ..   Male = col_double(),
##   ..   Female = col_double(),
##   ..   Intersex = col_double(),
##   ..   Total = col_double()
##   .. )
##  - attr(*, "problems")=<externalptr>
\end{verbatim}

\begin{Shaded}
\begin{Highlighting}[]
\CommentTok{\#Issue with city naming, not consistent through datasets}

\NormalTok{crops }\OtherTok{\textless{}{-}}\NormalTok{ crops }\SpecialCharTok{|\textgreater{}} 
  \FunctionTok{mutate}\NormalTok{(}
    \AttributeTok{County =}\NormalTok{ SubCounty,}
\NormalTok{  ) }\SpecialCharTok{|\textgreater{}} 
\NormalTok{  dplyr}\SpecialCharTok{::}\FunctionTok{select}\NormalTok{(}\SpecialCharTok{{-}}\NormalTok{SubCounty)}


\NormalTok{gender }\OtherTok{\textless{}{-}}\NormalTok{ gender }\SpecialCharTok{|\textgreater{}} 
  \FunctionTok{mutate}\NormalTok{(}
    \AttributeTok{County =} \FunctionTok{ifelse}\NormalTok{(County }\SpecialCharTok{==} \StringTok{"Total"}\NormalTok{, }\StringTok{"Kenya"}\NormalTok{, County)}
\NormalTok{  )}

\NormalTok{gender}\SpecialCharTok{$}\NormalTok{County }\OtherTok{\textless{}{-}} \FunctionTok{gsub}\NormalTok{(}\StringTok{" "}\NormalTok{, }\StringTok{""}\NormalTok{, gender}\SpecialCharTok{$}\NormalTok{County)}

\NormalTok{households}\SpecialCharTok{$}\NormalTok{County }\OtherTok{\textless{}{-}} \FunctionTok{trimws}\NormalTok{(households}\SpecialCharTok{$}\NormalTok{County)}
\FunctionTok{str}\NormalTok{(households)}
\end{Highlighting}
\end{Shaded}

\begin{verbatim}
## spc_tbl_ [48 x 4] (S3: spec_tbl_df/tbl_df/tbl/data.frame)
##  $ County              : chr [1:48] "Kenya" "Mombasa" "Kwale" "Kilifi" ...
##  $ Population          : num [1:48] 47213282 1190987 858748 1440958 314710 ...
##  $ NumberOfHouseholds  : num [1:48] 12143913 378422 173176 298472 68242 ...
##  $ AverageHouseholdSize: num [1:48] 3.9 3.1 5 4.8 4.6 3.7 3.5 5.9 6.1 6.9 ...
##  - attr(*, "spec")=
##   .. cols(
##   ..   County = col_character(),
##   ..   Population = col_double(),
##   ..   NumberOfHouseholds = col_double(),
##   ..   AverageHouseholdSize = col_double()
##   .. )
##  - attr(*, "problems")=<externalptr>
\end{verbatim}

\begin{Shaded}
\begin{Highlighting}[]
\FunctionTok{colnames}\NormalTok{(households)}
\end{Highlighting}
\end{Shaded}

\begin{verbatim}
## [1] "County"               "Population"           "NumberOfHouseholds"  
## [4] "AverageHouseholdSize"
\end{verbatim}

\begin{Shaded}
\begin{Highlighting}[]
\CommentTok{\#Function that tries to match the County names to each other by stringdist}

\NormalTok{find\_best\_match }\OtherTok{\textless{}{-}} \ControlFlowTok{function}\NormalTok{(county, reference\_counties) \{}
\NormalTok{  distances }\OtherTok{\textless{}{-}} \FunctionTok{stringdistmatrix}\NormalTok{(county, reference\_counties)}
\NormalTok{  best\_match\_index }\OtherTok{\textless{}{-}} \FunctionTok{which.min}\NormalTok{(distances)}
\NormalTok{  best\_match }\OtherTok{\textless{}{-}}\NormalTok{ reference\_counties[best\_match\_index]}
  \FunctionTok{return}\NormalTok{(best\_match)}
\NormalTok{\}}

\CommentTok{\#Function had an issue with only one value, the Nairobi city, probably due to two words in the name}
\NormalTok{crops}\SpecialCharTok{$}\NormalTok{County[crops}\SpecialCharTok{$}\NormalTok{County }\SpecialCharTok{==} \StringTok{"Nairobi"}\NormalTok{] }\OtherTok{\textless{}{-}} \StringTok{"Nairobi City"}
\NormalTok{crops}\SpecialCharTok{$}\NormalTok{best\_match }\OtherTok{\textless{}{-}} \FunctionTok{sapply}\NormalTok{(crops}\SpecialCharTok{$}\NormalTok{County, find\_best\_match, }\AttributeTok{reference\_counties =}\NormalTok{ gender}\SpecialCharTok{$}\NormalTok{County)}


\NormalTok{households}\SpecialCharTok{$}\NormalTok{best\_match }\OtherTok{\textless{}{-}} \FunctionTok{sapply}\NormalTok{(households}\SpecialCharTok{$}\NormalTok{County, find\_best\_match, }\AttributeTok{reference\_counties =}\NormalTok{ gender}\SpecialCharTok{$}\NormalTok{County)}


\NormalTok{merged\_dataset }\OtherTok{\textless{}{-}} \FunctionTok{left\_join}\NormalTok{(crops, households, }\AttributeTok{by =} \FunctionTok{c}\NormalTok{(}\StringTok{"best\_match"} \OtherTok{=} \StringTok{"County"}\NormalTok{)) }\SpecialCharTok{|\textgreater{}} 
  \FunctionTok{left\_join}\NormalTok{(gender, }\AttributeTok{by =} \FunctionTok{c}\NormalTok{(}\StringTok{"best\_match"} \OtherTok{=} \StringTok{"County"}\NormalTok{))}

\CommentTok{\# Clean up }
\NormalTok{merged\_dataset }\OtherTok{\textless{}{-}}\NormalTok{ merged\_dataset }\SpecialCharTok{|\textgreater{}} 
\NormalTok{  dplyr}\SpecialCharTok{::}\FunctionTok{select}\NormalTok{(}\SpecialCharTok{{-}}\NormalTok{best\_match, }\SpecialCharTok{{-}}\NormalTok{best\_match.y)}

\CommentTok{\#County in the first column}
\NormalTok{merged\_dataset }\OtherTok{\textless{}{-}}\NormalTok{ merged\_dataset }\SpecialCharTok{|\textgreater{}} 
\NormalTok{  dplyr}\SpecialCharTok{::}\FunctionTok{select}\NormalTok{(County, }\FunctionTok{everything}\NormalTok{())}
\end{Highlighting}
\end{Shaded}

\begin{Shaded}
\begin{Highlighting}[]
\CommentTok{\#Gender in the whole of Kenya}
\NormalTok{data\_kenya }\OtherTok{\textless{}{-}} \FunctionTok{filter}\NormalTok{(merged\_dataset, County }\SpecialCharTok{==} \StringTok{"Kenya"}\NormalTok{)}

\NormalTok{data\_kenya\_long }\OtherTok{\textless{}{-}} \FunctionTok{pivot\_longer}\NormalTok{(data\_kenya, }\AttributeTok{cols =} \FunctionTok{c}\NormalTok{(}\StringTok{"Male"}\NormalTok{, }\StringTok{"Female"}\NormalTok{, }\StringTok{"Intersex"}\NormalTok{), }\AttributeTok{names\_to =} \StringTok{"Gender"}\NormalTok{, }\AttributeTok{values\_to =} \StringTok{"Count"}\NormalTok{)}

\NormalTok{data\_kenya\_long }\SpecialCharTok{|\textgreater{}} 
  \FunctionTok{ggplot}\NormalTok{(}\FunctionTok{aes}\NormalTok{(}\AttributeTok{x =}\NormalTok{ Gender, }\AttributeTok{y =}\NormalTok{ Count)) }\SpecialCharTok{+}
  \FunctionTok{geom\_col}\NormalTok{()}
\end{Highlighting}
\end{Shaded}

\includegraphics{kenya_code_files/figure-latex/unnamed-chunk-1-1.pdf}

\begin{Shaded}
\begin{Highlighting}[]
\CommentTok{\#Farming production in the 5 biggest and smalles counties by production}
\NormalTok{smallest\_producers }\OtherTok{\textless{}{-}}\NormalTok{ merged\_dataset }\SpecialCharTok{|\textgreater{}} 
  \FunctionTok{arrange}\NormalTok{(Farming) }\SpecialCharTok{|\textgreater{}} 
  \FunctionTok{slice\_head}\NormalTok{(}\AttributeTok{n =} \DecValTok{5}\NormalTok{) }\SpecialCharTok{|\textgreater{}} 
  \FunctionTok{ggplot}\NormalTok{(}\FunctionTok{aes}\NormalTok{(}\AttributeTok{x =}\NormalTok{ County, }\AttributeTok{y =}\NormalTok{ Farming)) }\SpecialCharTok{+}
  \FunctionTok{geom\_col}\NormalTok{() }\SpecialCharTok{+}
  \FunctionTok{labs}\NormalTok{(}\AttributeTok{title =} \StringTok{"Top 5 Counties with Smallest Farming Values"}\NormalTok{,}
       \AttributeTok{x =} \StringTok{"County"}\NormalTok{,}
       \AttributeTok{y =} \StringTok{"Farming"}\NormalTok{) }\SpecialCharTok{+} 
  \FunctionTok{ylim}\NormalTok{(}\DecValTok{0}\NormalTok{, }\DecValTok{400000}\NormalTok{)}

\NormalTok{biggest\_producers }\OtherTok{\textless{}{-}}\NormalTok{ merged\_dataset }\SpecialCharTok{|\textgreater{}} 
  \FunctionTok{filter}\NormalTok{(County }\SpecialCharTok{!=} \StringTok{"Kenya"}\NormalTok{) }\SpecialCharTok{|\textgreater{}} 
  \FunctionTok{arrange}\NormalTok{(}\FunctionTok{desc}\NormalTok{(Farming)) }\SpecialCharTok{|\textgreater{}} 
  \FunctionTok{slice\_head}\NormalTok{(}\AttributeTok{n =} \DecValTok{5}\NormalTok{) }\SpecialCharTok{|\textgreater{}} 
  \FunctionTok{ggplot}\NormalTok{(}\FunctionTok{aes}\NormalTok{(}\AttributeTok{x =}\NormalTok{ County, }\AttributeTok{y =}\NormalTok{ Farming)) }\SpecialCharTok{+}
  \FunctionTok{geom\_col}\NormalTok{() }\SpecialCharTok{+}
  \FunctionTok{labs}\NormalTok{(}\AttributeTok{title =} \StringTok{"Top 5 Counties with Biggest Farming Values"}\NormalTok{,}
       \AttributeTok{x =} \StringTok{"County"}\NormalTok{,}
       \AttributeTok{y =} \StringTok{"Farming"}\NormalTok{) }\SpecialCharTok{+}
  \FunctionTok{ylim}\NormalTok{(}\DecValTok{0}\NormalTok{, }\DecValTok{400000}\NormalTok{)}

\FunctionTok{grid.arrange}\NormalTok{(smallest\_producers, biggest\_producers, }\AttributeTok{ncol =} \DecValTok{2}\NormalTok{)}
\end{Highlighting}
\end{Shaded}

\includegraphics{kenya_code_files/figure-latex/unnamed-chunk-1-2.pdf}

\begin{Shaded}
\begin{Highlighting}[]
\CommentTok{\#Population in the 5 biggest and smalles counties}

\NormalTok{smallest\_population }\OtherTok{\textless{}{-}}\NormalTok{ merged\_dataset }\SpecialCharTok{|\textgreater{}} 
  \FunctionTok{arrange}\NormalTok{(Population) }\SpecialCharTok{|\textgreater{}} 
  \FunctionTok{slice\_head}\NormalTok{(}\AttributeTok{n =} \DecValTok{5}\NormalTok{) }\SpecialCharTok{|\textgreater{}} 
  \FunctionTok{ggplot}\NormalTok{(}\FunctionTok{aes}\NormalTok{(}\AttributeTok{x =}\NormalTok{ County, }\AttributeTok{y =}\NormalTok{ Population)) }\SpecialCharTok{+}
  \FunctionTok{geom\_col}\NormalTok{() }\SpecialCharTok{+}
  \FunctionTok{labs}\NormalTok{(}\AttributeTok{title =} \StringTok{"Top 5 Counties with Smallest Population"}\NormalTok{,}
       \AttributeTok{x =} \StringTok{"County"}\NormalTok{,}
       \AttributeTok{y =} \StringTok{"Population"}\NormalTok{) }\SpecialCharTok{+} 
  \FunctionTok{ylim}\NormalTok{(}\DecValTok{0}\NormalTok{, }\DecValTok{3000000}\NormalTok{)}

\NormalTok{biggest\_population }\OtherTok{\textless{}{-}}\NormalTok{ merged\_dataset }\SpecialCharTok{|\textgreater{}}
  \FunctionTok{filter}\NormalTok{(County }\SpecialCharTok{!=} \StringTok{"Kenya"}\NormalTok{) }\SpecialCharTok{|\textgreater{}} 
  \FunctionTok{arrange}\NormalTok{(}\FunctionTok{desc}\NormalTok{(Population)) }\SpecialCharTok{|\textgreater{}} 
  \FunctionTok{slice\_head}\NormalTok{(}\AttributeTok{n =} \DecValTok{5}\NormalTok{) }\SpecialCharTok{|\textgreater{}} 
  \FunctionTok{ggplot}\NormalTok{(}\FunctionTok{aes}\NormalTok{(}\AttributeTok{x =}\NormalTok{ County, }\AttributeTok{y =}\NormalTok{ Population)) }\SpecialCharTok{+}
  \FunctionTok{geom\_col}\NormalTok{() }\SpecialCharTok{+}
  \FunctionTok{labs}\NormalTok{(}\AttributeTok{title =} \StringTok{"Top 5 Counties with Biggest Population"}\NormalTok{,}
       \AttributeTok{x =} \StringTok{"County"}\NormalTok{,}
       \AttributeTok{y =} \StringTok{"Population"}\NormalTok{) }\SpecialCharTok{+} 
  \FunctionTok{ylim}\NormalTok{(}\DecValTok{0}\NormalTok{, }\DecValTok{3000000}\NormalTok{)}

\FunctionTok{grid.arrange}\NormalTok{(smallest\_population, biggest\_population, }\AttributeTok{ncol =} \DecValTok{2}\NormalTok{)}
\end{Highlighting}
\end{Shaded}

\includegraphics{kenya_code_files/figure-latex/unnamed-chunk-2-1.pdf}

\begin{Shaded}
\begin{Highlighting}[]
\CommentTok{\#Means}
\NormalTok{merged\_dataset }\SpecialCharTok{\%\textgreater{}\%}
  \FunctionTok{summarise}\NormalTok{(}
    \AttributeTok{Population =} \FunctionTok{mean}\NormalTok{(Population),}
    \AttributeTok{Farming =} \FunctionTok{mean}\NormalTok{(Farming),}
    \StringTok{"N of households"} \OtherTok{=} \FunctionTok{mean}\NormalTok{(NumberOfHouseholds)}
\NormalTok{  ) }\SpecialCharTok{\%\textgreater{}\%}
  \FunctionTok{pivot\_longer}\NormalTok{(}\FunctionTok{everything}\NormalTok{(), }\AttributeTok{names\_to =} \StringTok{"Variable"}\NormalTok{, }\AttributeTok{values\_to =} \StringTok{"Mean"}\NormalTok{) }\SpecialCharTok{\%\textgreater{}\%}
  \FunctionTok{mutate}\NormalTok{(}
    \AttributeTok{Mean =} \FunctionTok{round}\NormalTok{(Mean, }\DecValTok{0}\NormalTok{)}
\NormalTok{  ) }\SpecialCharTok{|\textgreater{}} 
  \FunctionTok{gt}\NormalTok{() }\SpecialCharTok{\%\textgreater{}\%}
  \FunctionTok{tab\_header}\NormalTok{(}
    \AttributeTok{title =} \FunctionTok{md}\NormalTok{(}\StringTok{"**Descriptive statistics**"}\NormalTok{),}
    \AttributeTok{subtitle =} \StringTok{"Kenya census 2019"}
\NormalTok{  ) }\SpecialCharTok{\%\textgreater{}\%}
  \FunctionTok{fmt\_number}\NormalTok{(}
    \AttributeTok{columns =} \FunctionTok{vars}\NormalTok{(Variable)}
\NormalTok{  )}
\end{Highlighting}
\end{Shaded}

\begin{longtable}{lr}
\caption*{
{\large \textbf{Descriptive statistics}} \\ 
{\small Kenya census 2019}
} \\ 
\toprule
Variable & Mean \\ 
\midrule\addlinespace[2.5pt]
Population & 1967220 \\ 
Farming & 264759 \\ 
N of households & 505996 \\ 
\bottomrule
\end{longtable}

\begin{Shaded}
\begin{Highlighting}[]
\CommentTok{\#Standard deviations}
\NormalTok{merged\_dataset }\SpecialCharTok{|\textgreater{}} 
  \FunctionTok{summarise}\NormalTok{(}
    \AttributeTok{Population =} \FunctionTok{sd}\NormalTok{(Population),}
    \AttributeTok{Farming =} \FunctionTok{sd}\NormalTok{(Farming),}
    \StringTok{"N of households"} \OtherTok{=} \FunctionTok{sd}\NormalTok{(NumberOfHouseholds)}
\NormalTok{  ) }\SpecialCharTok{|\textgreater{}} 
  \FunctionTok{pivot\_longer}\NormalTok{(}\FunctionTok{everything}\NormalTok{(), }\AttributeTok{names\_to =} \StringTok{"Variable"}\NormalTok{, }\AttributeTok{values\_to =} \StringTok{"SD"}\NormalTok{) }\SpecialCharTok{|\textgreater{}} 
  \FunctionTok{mutate}\NormalTok{(}
    \AttributeTok{SD =} \FunctionTok{round}\NormalTok{(SD, }\DecValTok{0}\NormalTok{)}
\NormalTok{  ) }\SpecialCharTok{|\textgreater{}}
  \FunctionTok{gt}\NormalTok{() }\SpecialCharTok{|\textgreater{}} 
  \FunctionTok{tab\_header}\NormalTok{(}
    \AttributeTok{title =} \FunctionTok{md}\NormalTok{(}\StringTok{"**Descriptive statistics**"}\NormalTok{),}
    \AttributeTok{subtitle =} \StringTok{"Kenya census 2019"}
\NormalTok{  ) }\SpecialCharTok{|\textgreater{}} 
  \FunctionTok{fmt\_number}\NormalTok{(}
    \AttributeTok{columns =} \FunctionTok{c}\NormalTok{(Variable)}
\NormalTok{  )}
\end{Highlighting}
\end{Shaded}

\begin{longtable}{lr}
\caption*{
{\large \textbf{Descriptive statistics}} \\ 
{\small Kenya census 2019}
} \\ 
\toprule
Variable & SD \\ 
\midrule\addlinespace[2.5pt]
Population & 6704047 \\ 
Farming & 900960 \\ 
N of households & 1731138 \\ 
\bottomrule
\end{longtable}

\begin{Shaded}
\begin{Highlighting}[]
\CommentTok{\#Outliers with boxplots}

\DocumentationTok{\#\#Population}
\NormalTok{merged\_dataset }\SpecialCharTok{|\textgreater{}} 
  \FunctionTok{filter}\NormalTok{(County }\SpecialCharTok{!=} \StringTok{"Kenya"}\NormalTok{) }\SpecialCharTok{|\textgreater{}} 
  \FunctionTok{ggplot}\NormalTok{(}\FunctionTok{aes}\NormalTok{(}\AttributeTok{x =} \DecValTok{1}\NormalTok{, }\AttributeTok{y =}\NormalTok{ Population)) }\SpecialCharTok{+}
  \FunctionTok{geom\_boxplot}\NormalTok{() }\SpecialCharTok{+}
  \FunctionTok{geom\_text}\NormalTok{(}\FunctionTok{aes}\NormalTok{(}\AttributeTok{label =} \FunctionTok{ifelse}\NormalTok{(Population }\SpecialCharTok{\textless{}} \FunctionTok{quantile}\NormalTok{(Population, }\FloatTok{0.25}\NormalTok{) }\SpecialCharTok{{-}} \FloatTok{1.5} \SpecialCharTok{*} \FunctionTok{IQR}\NormalTok{(Population) }\SpecialCharTok{|} 
\NormalTok{                                  Population }\SpecialCharTok{\textgreater{}} \FunctionTok{quantile}\NormalTok{(Population, }\FloatTok{0.75}\NormalTok{) }\SpecialCharTok{+} \FloatTok{1.5} \SpecialCharTok{*} \FunctionTok{IQR}\NormalTok{(Population), County, }\StringTok{""}\NormalTok{)),}
            \AttributeTok{size =} \DecValTok{5}\NormalTok{)}
\end{Highlighting}
\end{Shaded}

\includegraphics{kenya_code_files/figure-latex/Descriptives-1.pdf}

\begin{Shaded}
\begin{Highlighting}[]
\DocumentationTok{\#\#Farming}
\NormalTok{merged\_dataset }\SpecialCharTok{|\textgreater{}} 
  \FunctionTok{filter}\NormalTok{(County }\SpecialCharTok{!=} \StringTok{"Kenya"}\NormalTok{) }\SpecialCharTok{|\textgreater{}} 
  \FunctionTok{ggplot}\NormalTok{(}\FunctionTok{aes}\NormalTok{(}\AttributeTok{x =} \DecValTok{1}\NormalTok{, }\AttributeTok{y =}\NormalTok{ Farming)) }\SpecialCharTok{+}
  \FunctionTok{geom\_boxplot}\NormalTok{() }\SpecialCharTok{+}
  \FunctionTok{geom\_text}\NormalTok{(}\FunctionTok{aes}\NormalTok{(}\AttributeTok{label =} \FunctionTok{ifelse}\NormalTok{(Farming }\SpecialCharTok{\textless{}} \FunctionTok{quantile}\NormalTok{(Farming, }\FloatTok{0.25}\NormalTok{) }\SpecialCharTok{{-}} \FloatTok{1.5} \SpecialCharTok{*} \FunctionTok{IQR}\NormalTok{(Farming) }\SpecialCharTok{|} 
\NormalTok{                                  Farming }\SpecialCharTok{\textgreater{}} \FunctionTok{quantile}\NormalTok{(Farming, }\FloatTok{0.75}\NormalTok{) }\SpecialCharTok{+} \FloatTok{1.5} \SpecialCharTok{*} \FunctionTok{IQR}\NormalTok{(Farming), County, }\StringTok{""}\NormalTok{)),}
            \AttributeTok{size =} \DecValTok{5}\NormalTok{)}
\end{Highlighting}
\end{Shaded}

\includegraphics{kenya_code_files/figure-latex/Descriptives-2.pdf}

\begin{Shaded}
\begin{Highlighting}[]
\DocumentationTok{\#\#Households}
\NormalTok{merged\_dataset }\SpecialCharTok{|\textgreater{}}
  \FunctionTok{filter}\NormalTok{(County }\SpecialCharTok{!=} \StringTok{"Kenya"}\NormalTok{) }\SpecialCharTok{|\textgreater{}} 
  \FunctionTok{ggplot}\NormalTok{(}\FunctionTok{aes}\NormalTok{(}\AttributeTok{x =} \DecValTok{1}\NormalTok{, }\AttributeTok{y =}\NormalTok{ NumberOfHouseholds)) }\SpecialCharTok{+}
  \FunctionTok{geom\_boxplot}\NormalTok{() }\SpecialCharTok{+}
  \FunctionTok{geom\_text}\NormalTok{(}\FunctionTok{aes}\NormalTok{(}\AttributeTok{label =} \FunctionTok{ifelse}\NormalTok{(NumberOfHouseholds }\SpecialCharTok{\textless{}} \FunctionTok{quantile}\NormalTok{(NumberOfHouseholds, }\FloatTok{0.25}\NormalTok{) }\SpecialCharTok{{-}} \FloatTok{1.5} \SpecialCharTok{*} \FunctionTok{IQR}\NormalTok{(NumberOfHouseholds) }\SpecialCharTok{|} 
\NormalTok{                                  NumberOfHouseholds }\SpecialCharTok{\textgreater{}} \FunctionTok{quantile}\NormalTok{(NumberOfHouseholds, }\FloatTok{0.75}\NormalTok{) }\SpecialCharTok{+} \FloatTok{1.5} \SpecialCharTok{*} \FunctionTok{IQR}\NormalTok{(NumberOfHouseholds), County, }\StringTok{""}\NormalTok{)),}
            \AttributeTok{size =} \DecValTok{5}\NormalTok{)}
\end{Highlighting}
\end{Shaded}

\includegraphics{kenya_code_files/figure-latex/Descriptives-3.pdf}

\begin{Shaded}
\begin{Highlighting}[]
\CommentTok{\#Conclusion: Even though there are three outliers, I will not remove them because they represent a natural occurence, where people congregate in urban areas, and the outliers are the biggest cities in Kenya.}

\CommentTok{\#Plant production distribution}

\NormalTok{plants }\OtherTok{\textless{}{-}} \FunctionTok{c}\NormalTok{(}\StringTok{"Tea"}\NormalTok{, }\StringTok{"Coffee"}\NormalTok{, }\StringTok{"Avocado"}\NormalTok{, }\StringTok{"Citrus"}\NormalTok{, }\StringTok{"Mango"}\NormalTok{, }\StringTok{"Coconut"}\NormalTok{, }\StringTok{"Macadamia"}\NormalTok{, }\StringTok{"Cashew Nut"}\NormalTok{)}

\NormalTok{plant\_colors }\OtherTok{\textless{}{-}} \FunctionTok{c}\NormalTok{(}
  \AttributeTok{Tea =} \StringTok{"darkgreen"}\NormalTok{,}
  \AttributeTok{Coffee =} \StringTok{"saddlebrown"}\NormalTok{,}
  \AttributeTok{Avocado =} \StringTok{"limegreen"}\NormalTok{,}
  \AttributeTok{Citrus =} \StringTok{"yellow"}\NormalTok{,}
  \AttributeTok{Mango =} \StringTok{"darkorange"}\NormalTok{,}
  \AttributeTok{Coconut =} \StringTok{"\#d0d1e6"}\NormalTok{,}
  \AttributeTok{Macadamia =} \StringTok{"darkolivegreen"}\NormalTok{,}
  \StringTok{"Cashew Nut"} \OtherTok{=} \StringTok{"goldenrod"}
\NormalTok{)}

\NormalTok{merged\_dataset }\SpecialCharTok{|\textgreater{}} 
  \FunctionTok{filter}\NormalTok{(County }\SpecialCharTok{==} \StringTok{"Kenya"}\NormalTok{) }\SpecialCharTok{|\textgreater{}} 
  \FunctionTok{pivot\_longer}\NormalTok{(}\AttributeTok{cols =}\NormalTok{ plants, }\AttributeTok{names\_to =} \StringTok{"Crop"}\NormalTok{, }\AttributeTok{values\_to =} \StringTok{"Production"}\NormalTok{) }\SpecialCharTok{|\textgreater{}} 
  \FunctionTok{ggplot}\NormalTok{(}\FunctionTok{aes}\NormalTok{(}\AttributeTok{x =} \FunctionTok{fct\_reorder}\NormalTok{(Crop, Production), }\AttributeTok{y =}\NormalTok{ Production)) }\SpecialCharTok{+}
  \FunctionTok{geom\_bar}\NormalTok{(}\AttributeTok{stat =} \StringTok{"identity"}\NormalTok{, }\AttributeTok{fill =}\NormalTok{ plant\_colors) }\SpecialCharTok{+}
  \FunctionTok{coord\_flip}\NormalTok{() }\SpecialCharTok{+}
  \FunctionTok{theme\_minimal}\NormalTok{() }\SpecialCharTok{+}
  \FunctionTok{labs}\NormalTok{(}
    \AttributeTok{title =} \StringTok{"Plant production in Kenya"}\NormalTok{,}
    \AttributeTok{x =} \StringTok{"Crop"}\NormalTok{,}
    \AttributeTok{y =} \StringTok{"Production"}
\NormalTok{  )}
\end{Highlighting}
\end{Shaded}

\includegraphics{kenya_code_files/figure-latex/Descriptives-4.pdf}

\begin{Shaded}
\begin{Highlighting}[]
\CommentTok{\#Linear regression}
\DocumentationTok{\#\#H1: Does the population affect the farming production?}
\DocumentationTok{\#\#H2: Is there a positive correlation between Tea and Coffee production?}


\CommentTok{\#Mediation}
\DocumentationTok{\#\#H3: Farming is a mediator between population and number of households}
\end{Highlighting}
\end{Shaded}

\begin{Shaded}
\begin{Highlighting}[]
\CommentTok{\#Does the population affect the farming production?}
\NormalTok{counties\_data }\OtherTok{\textless{}{-}}\NormalTok{ merged\_dataset }\SpecialCharTok{|\textgreater{}} 
  \FunctionTok{filter}\NormalTok{(County }\SpecialCharTok{!=} \StringTok{"Kenya"}\NormalTok{)}

\FunctionTok{cor}\NormalTok{(counties\_data}\SpecialCharTok{$}\NormalTok{Population, counties\_data}\SpecialCharTok{$}\NormalTok{Farming)}
\end{Highlighting}
\end{Shaded}

\begin{verbatim}
## [1] 0.3836631
\end{verbatim}

\begin{Shaded}
\begin{Highlighting}[]
\NormalTok{counties\_data }\SpecialCharTok{|\textgreater{}} 
  \FunctionTok{ggplot}\NormalTok{(}\FunctionTok{aes}\NormalTok{(}\AttributeTok{x =}\NormalTok{ Population, }\AttributeTok{y =}\NormalTok{ Farming)) }\SpecialCharTok{+}
  \FunctionTok{geom\_point}\NormalTok{() }\SpecialCharTok{+}
  \FunctionTok{geom\_smooth}\NormalTok{(}\AttributeTok{method =} \StringTok{"lm"}\NormalTok{, }\AttributeTok{se =} \ConstantTok{FALSE}\NormalTok{) }\SpecialCharTok{+}
  \FunctionTok{labs}\NormalTok{(}
    \AttributeTok{title =} \StringTok{"Population and Farming Production"}\NormalTok{,}
    \AttributeTok{x =} \StringTok{"Population"}\NormalTok{,}
    \AttributeTok{y =} \StringTok{"Farming"}
\NormalTok{  ) }\SpecialCharTok{+}
  \FunctionTok{theme\_minimal}\NormalTok{()}
\end{Highlighting}
\end{Shaded}

\includegraphics{kenya_code_files/figure-latex/H1, linear regression model 1-1.pdf}

\begin{Shaded}
\begin{Highlighting}[]
\CommentTok{\#Assumption 1.1: Normality}



\FunctionTok{shapiro.test}\NormalTok{(counties\_data}\SpecialCharTok{$}\NormalTok{Population)}
\end{Highlighting}
\end{Shaded}

\begin{verbatim}
## 
##  Shapiro-Wilk normality test
## 
## data:  counties_data$Population
## W = 0.76621, p-value = 3.004e-07
\end{verbatim}

\begin{Shaded}
\begin{Highlighting}[]
\FunctionTok{hist}\NormalTok{(counties\_data}\SpecialCharTok{$}\NormalTok{Population)}
\end{Highlighting}
\end{Shaded}

\includegraphics{kenya_code_files/figure-latex/H1, linear regression model 1-2.pdf}

\begin{Shaded}
\begin{Highlighting}[]
\FunctionTok{shapiro.test}\NormalTok{(counties\_data}\SpecialCharTok{$}\NormalTok{Farming)}
\end{Highlighting}
\end{Shaded}

\begin{verbatim}
## 
##  Shapiro-Wilk normality test
## 
## data:  counties_data$Farming
## W = 0.96723, p-value = 0.2073
\end{verbatim}

\begin{Shaded}
\begin{Highlighting}[]
\FunctionTok{hist}\NormalTok{(counties\_data}\SpecialCharTok{$}\NormalTok{Farming)}
\end{Highlighting}
\end{Shaded}

\includegraphics{kenya_code_files/figure-latex/H1, linear regression model 1-3.pdf}

\begin{Shaded}
\begin{Highlighting}[]
\DocumentationTok{\#\#Nairobi significantly alters the normality}

\NormalTok{pop\_farming }\OtherTok{\textless{}{-}} \FunctionTok{lm}\NormalTok{(Farming }\SpecialCharTok{\textasciitilde{}}\NormalTok{ Population, }\AttributeTok{data =}\NormalTok{ counties\_data)}
\FunctionTok{glance}\NormalTok{(pop\_farming)}
\end{Highlighting}
\end{Shaded}

\begin{verbatim}
## # A tibble: 1 x 12
##   r.squared adj.r.squared  sigma statistic p.value    df logLik   AIC   BIC
##       <dbl>         <dbl>  <dbl>     <dbl>   <dbl> <dbl>  <dbl> <dbl> <dbl>
## 1     0.147         0.128 72960.      7.77 0.00776     1  -592. 1190. 1195.
## # i 3 more variables: deviance <dbl>, df.residual <int>, nobs <int>
\end{verbatim}

\begin{Shaded}
\begin{Highlighting}[]
\CommentTok{\#Assumption 1.2: Linearity}

\FunctionTok{raintest}\NormalTok{(pop\_farming)}
\end{Highlighting}
\end{Shaded}

\begin{verbatim}
## 
##  Rainbow test
## 
## data:  pop_farming
## Rain = 4.9462, df1 = 24, df2 = 21, p-value = 0.0002283
\end{verbatim}

\begin{Shaded}
\begin{Highlighting}[]
\DocumentationTok{\#\#Shows nonlinearity}

\CommentTok{\#Assumption 1.3: Homoscedasticity}

\FunctionTok{bptest}\NormalTok{(pop\_farming)}
\end{Highlighting}
\end{Shaded}

\begin{verbatim}
## 
##  studentized Breusch-Pagan test
## 
## data:  pop_farming
## BP = 27.443, df = 1, p-value = 1.618e-07
\end{verbatim}

\begin{Shaded}
\begin{Highlighting}[]
\DocumentationTok{\#\#Shows heteroscedasticity}

\CommentTok{\#Assumption 1.4: Normality of residuals}

\FunctionTok{shapiro.test}\NormalTok{(pop\_farming}\SpecialCharTok{$}\NormalTok{residuals)}
\end{Highlighting}
\end{Shaded}

\begin{verbatim}
## 
##  Shapiro-Wilk normality test
## 
## data:  pop_farming$residuals
## W = 0.96207, p-value = 0.13
\end{verbatim}

\begin{Shaded}
\begin{Highlighting}[]
\DocumentationTok{\#\#Normality of residuals met}

\CommentTok{\#Overall the model does not work with Nairobi in it}
\end{Highlighting}
\end{Shaded}

\begin{Shaded}
\begin{Highlighting}[]
\CommentTok{\#Assumption 2.1: Normality}

\NormalTok{counties\_data\_no\_capital }\OtherTok{\textless{}{-}}\NormalTok{ counties\_data }\SpecialCharTok{|\textgreater{}} 
  \FunctionTok{filter}\NormalTok{(County }\SpecialCharTok{!=} \StringTok{"Nairobi City"}\NormalTok{)}

\FunctionTok{shapiro.test}\NormalTok{(counties\_data\_no\_capital}\SpecialCharTok{$}\NormalTok{Population)}
\end{Highlighting}
\end{Shaded}

\begin{verbatim}
## 
##  Shapiro-Wilk normality test
## 
## data:  counties_data_no_capital$Population
## W = 0.94095, p-value = 0.02124
\end{verbatim}

\begin{Shaded}
\begin{Highlighting}[]
\FunctionTok{hist}\NormalTok{(counties\_data\_no\_capital}\SpecialCharTok{$}\NormalTok{Population)}
\end{Highlighting}
\end{Shaded}

\includegraphics{kenya_code_files/figure-latex/H1, linear regression model 2-1.pdf}

\begin{Shaded}
\begin{Highlighting}[]
\DocumentationTok{\#\#Removing Nairobi results in a normal population distribution}

\NormalTok{pop\_farming\_no\_capital }\OtherTok{\textless{}{-}} \FunctionTok{lm}\NormalTok{(Farming }\SpecialCharTok{\textasciitilde{}}\NormalTok{ Population, }\AttributeTok{data =}\NormalTok{ counties\_data\_no\_capital)}
\FunctionTok{glance}\NormalTok{(pop\_farming\_no\_capital)}
\end{Highlighting}
\end{Shaded}

\begin{verbatim}
## # A tibble: 1 x 12
##   r.squared adj.r.squared  sigma statistic  p.value    df logLik   AIC   BIC
##       <dbl>         <dbl>  <dbl>     <dbl>    <dbl> <dbl>  <dbl> <dbl> <dbl>
## 1     0.606         0.597 49204.      67.5 1.96e-10     1  -561. 1128. 1134.
## # i 3 more variables: deviance <dbl>, df.residual <int>, nobs <int>
\end{verbatim}

\begin{Shaded}
\begin{Highlighting}[]
\CommentTok{\#Assumption 2.2: Linearity}

\FunctionTok{raintest}\NormalTok{(pop\_farming\_no\_capital)}
\end{Highlighting}
\end{Shaded}

\begin{verbatim}
## 
##  Rainbow test
## 
## data:  pop_farming_no_capital
## Rain = 1.7883, df1 = 23, df2 = 21, p-value = 0.0926
\end{verbatim}

\begin{Shaded}
\begin{Highlighting}[]
\DocumentationTok{\#\#Shows linearity}

\CommentTok{\#Assumption 2.3: Homoscedasticity}

\FunctionTok{bptest}\NormalTok{(pop\_farming\_no\_capital)}
\end{Highlighting}
\end{Shaded}

\begin{verbatim}
## 
##  studentized Breusch-Pagan test
## 
## data:  pop_farming_no_capital
## BP = 6.2772, df = 1, p-value = 0.01223
\end{verbatim}

\begin{Shaded}
\begin{Highlighting}[]
\DocumentationTok{\#\#Still shows heteroscedasticity}

\CommentTok{\#Assumption 2.4: Normality of residuals}

\FunctionTok{shapiro.test}\NormalTok{(pop\_farming\_no\_capital}\SpecialCharTok{$}\NormalTok{residuals)}
\end{Highlighting}
\end{Shaded}

\begin{verbatim}
## 
##  Shapiro-Wilk normality test
## 
## data:  pop_farming_no_capital$residuals
## W = 0.94854, p-value = 0.04122
\end{verbatim}

\begin{Shaded}
\begin{Highlighting}[]
\DocumentationTok{\#\#Normality of residuals not met}

\DocumentationTok{\#\#This model is still not sufficient, but it massively improves explanatory or predictive power, as can be seen by the R2 value.}
\end{Highlighting}
\end{Shaded}

\begin{Shaded}
\begin{Highlighting}[]
\CommentTok{\#Final model}

\DocumentationTok{\#\#Using the data without Nairobi so linearity is met}

\NormalTok{bc }\OtherTok{\textless{}{-}} \FunctionTok{boxcox}\NormalTok{(Farming }\SpecialCharTok{\textasciitilde{}}\NormalTok{ Population, }\AttributeTok{data =}\NormalTok{ counties\_data\_no\_capital)}
\end{Highlighting}
\end{Shaded}

\includegraphics{kenya_code_files/figure-latex/H1, robust regression-1.pdf}

\begin{Shaded}
\begin{Highlighting}[]
\NormalTok{lambda }\OtherTok{\textless{}{-}}\NormalTok{ bc}\SpecialCharTok{$}\NormalTok{x[}\FunctionTok{which.max}\NormalTok{(bc}\SpecialCharTok{$}\NormalTok{y)]}


\NormalTok{box\_cox\_model }\OtherTok{\textless{}{-}} \FunctionTok{lmrob}\NormalTok{(((Farming}\SpecialCharTok{\^{}}\NormalTok{lambda}\DecValTok{{-}1}\NormalTok{)}\SpecialCharTok{/}\NormalTok{lambda) }\SpecialCharTok{\textasciitilde{}}\NormalTok{ Population, }\AttributeTok{data =}\NormalTok{ counties\_data\_no\_capital)}

\FunctionTok{qqnorm}\NormalTok{(box\_cox\_model}\SpecialCharTok{$}\NormalTok{residuals)}
\end{Highlighting}
\end{Shaded}

\includegraphics{kenya_code_files/figure-latex/H1, robust regression-2.pdf}

\begin{Shaded}
\begin{Highlighting}[]
\CommentTok{\#Box{-}Cox transformation}



\CommentTok{\#Huber regression}
\NormalTok{box\_cox\_model }\OtherTok{\textless{}{-}} \FunctionTok{lmrob}\NormalTok{(Farming }\SpecialCharTok{\textasciitilde{}}\NormalTok{ Population, }\AttributeTok{data =}\NormalTok{ counties\_data\_no\_capital)}
\FunctionTok{summary}\NormalTok{(box\_cox\_model)}
\end{Highlighting}
\end{Shaded}

\begin{verbatim}
## 
## Call:
## lmrob(formula = Farming ~ Population, data = counties_data_no_capital)
##  \--> method = "MM"
## Residuals:
##     Min      1Q  Median      3Q     Max 
## -175556  -32191    1443   20609   64752 
## 
## Coefficients:
##              Estimate Std. Error t value Pr(>|t|)    
## (Intercept) 3.223e+03  1.132e+04   0.285    0.777    
## Population  1.552e-01  1.578e-02   9.835 1.11e-12 ***
## ---
## Signif. codes:  0 '***' 0.001 '**' 0.01 '*' 0.05 '.' 0.1 ' ' 1
## 
## Robust residual standard error: 33830 
## Multiple R-squared:  0.7623, Adjusted R-squared:  0.7569 
## Convergence in 15 IRWLS iterations
## 
## Robustness weights: 
##  2 observations c(1,22) are outliers with |weight| = 0 ( < 0.0022); 
##  3 weights are ~= 1. The remaining 41 ones are summarized as
##    Min. 1st Qu.  Median    Mean 3rd Qu.    Max. 
##  0.5052  0.8604  0.9367  0.8874  0.9861  0.9977 
## Algorithmic parameters: 
##        tuning.chi                bb        tuning.psi        refine.tol 
##         1.548e+00         5.000e-01         4.685e+00         1.000e-07 
##           rel.tol         scale.tol         solve.tol          zero.tol 
##         1.000e-07         1.000e-10         1.000e-07         1.000e-10 
##       eps.outlier             eps.x warn.limit.reject warn.limit.meanrw 
##         2.174e-03         4.371e-06         5.000e-01         5.000e-01 
##      nResample         max.it       best.r.s       k.fast.s          k.max 
##            500             50              2              1            200 
##    maxit.scale      trace.lev            mts     compute.rd fast.s.large.n 
##            200              0           1000              0           2000 
##                   psi           subsampling                   cov 
##            "bisquare"         "nonsingular"         ".vcov.avar1" 
## compute.outlier.stats 
##                  "SM" 
## seed : int(0)
\end{verbatim}

\begin{Shaded}
\begin{Highlighting}[]
\CommentTok{\#Assumption 3.1: Normality of residuals}

\FunctionTok{shapiro.test}\NormalTok{(box\_cox\_model}\SpecialCharTok{$}\NormalTok{residuals)}
\end{Highlighting}
\end{Shaded}

\begin{verbatim}
## 
##  Shapiro-Wilk normality test
## 
## data:  box_cox_model$residuals
## W = 0.88047, p-value = 0.0002115
\end{verbatim}

\begin{Shaded}
\begin{Highlighting}[]
\DocumentationTok{\#\#Assumption not met}

\CommentTok{\#Assumption 3.2: Linearity}

\FunctionTok{raintest}\NormalTok{(box\_cox\_model)}
\end{Highlighting}
\end{Shaded}

\begin{verbatim}
## 
##  Rainbow test
## 
## data:  box_cox_model
## Rain = 1.7883, df1 = 23, df2 = 21, p-value = 0.0926
\end{verbatim}

\begin{Shaded}
\begin{Highlighting}[]
\CommentTok{\#Assumption met}

\CommentTok{\#Assumption 3.3: Homoscedasticity}

\FunctionTok{bptest}\NormalTok{(box\_cox\_model)}
\end{Highlighting}
\end{Shaded}

\begin{verbatim}
## 
##  studentized Breusch-Pagan test
## 
## data:  box_cox_model
## BP = 6.2772, df = 1, p-value = 0.01223
\end{verbatim}

\begin{Shaded}
\begin{Highlighting}[]
\CommentTok{\#Assumption not met}

\CommentTok{\#Furthermore, robust regression is not suitable for less than 100 observations}
\end{Highlighting}
\end{Shaded}

In conclusion, there are too many issues with the data to create a
reliable regression between population and farming, especially because
of the issue of nonlinearity.

\begin{Shaded}
\begin{Highlighting}[]
\DocumentationTok{\#\#H2: Is there a positive correlation between Tea and Coffee production?}

\FunctionTok{cor}\NormalTok{(counties\_data\_no\_capital}\SpecialCharTok{$}\NormalTok{Tea, counties\_data\_no\_capital}\SpecialCharTok{$}\NormalTok{Coffee, }\AttributeTok{use =} \StringTok{"pairwise.complete.obs"}\NormalTok{)}
\end{Highlighting}
\end{Shaded}

\begin{verbatim}
## [1] 0.4166591
\end{verbatim}

\begin{Shaded}
\begin{Highlighting}[]
\NormalTok{counties\_data\_no\_capital }\SpecialCharTok{|\textgreater{}} 
  \FunctionTok{ggplot}\NormalTok{(}\FunctionTok{aes}\NormalTok{(}\AttributeTok{x =}\NormalTok{ Tea, }\AttributeTok{y =}\NormalTok{ Coffee)) }\SpecialCharTok{+}
  \FunctionTok{geom\_point}\NormalTok{() }\SpecialCharTok{+}
  \FunctionTok{geom\_smooth}\NormalTok{(}\AttributeTok{method =} \StringTok{"lm"}\NormalTok{, }\AttributeTok{se =} \ConstantTok{FALSE}\NormalTok{) }\SpecialCharTok{+}
  \FunctionTok{labs}\NormalTok{(}
    \AttributeTok{title =} \StringTok{"Tea and Coffee Production"}\NormalTok{,}
    \AttributeTok{x =} \StringTok{"Tea"}\NormalTok{,}
    \AttributeTok{y =} \StringTok{"Coffee"}
\NormalTok{  ) }\SpecialCharTok{+}
  \FunctionTok{theme\_minimal}\NormalTok{()}
\end{Highlighting}
\end{Shaded}

\includegraphics{kenya_code_files/figure-latex/H2-1.pdf}

\begin{Shaded}
\begin{Highlighting}[]
\NormalTok{tea\_model }\OtherTok{\textless{}{-}} \FunctionTok{lm}\NormalTok{(Tea }\SpecialCharTok{\textasciitilde{}}\NormalTok{ Coffee, }\AttributeTok{data =}\NormalTok{ counties\_data\_no\_capital)}

\CommentTok{\#Assumption 4.1: Normality}

\FunctionTok{shapiro.test}\NormalTok{(counties\_data\_no\_capital}\SpecialCharTok{$}\NormalTok{Tea)}
\end{Highlighting}
\end{Shaded}

\begin{verbatim}
## 
##  Shapiro-Wilk normality test
## 
## data:  counties_data_no_capital$Tea
## W = 0.89963, p-value = 0.0479
\end{verbatim}

\begin{Shaded}
\begin{Highlighting}[]
\DocumentationTok{\#\#Normality not met}

\NormalTok{counties\_data\_no\_capital\_log }\OtherTok{\textless{}{-}}\NormalTok{ counties\_data\_no\_capital }\SpecialCharTok{|\textgreater{}} 
  \FunctionTok{mutate}\NormalTok{(}\AttributeTok{Tea =} \FunctionTok{log}\NormalTok{(Tea),}
         \AttributeTok{Coffee =} \FunctionTok{log}\NormalTok{(Coffee))}

\FunctionTok{shapiro.test}\NormalTok{(counties\_data\_no\_capital\_log}\SpecialCharTok{$}\NormalTok{Tea)}
\end{Highlighting}
\end{Shaded}

\begin{verbatim}
## 
##  Shapiro-Wilk normality test
## 
## data:  counties_data_no_capital_log$Tea
## W = 0.91342, p-value = 0.08557
\end{verbatim}

\begin{Shaded}
\begin{Highlighting}[]
\CommentTok{\#Normality met}

\NormalTok{tea\_model }\OtherTok{\textless{}{-}} \FunctionTok{lm}\NormalTok{(Tea }\SpecialCharTok{\textasciitilde{}}\NormalTok{ Coffee, }\AttributeTok{data =}\NormalTok{ counties\_data\_no\_capital\_log)}

\CommentTok{\#Assumption 4.2: Linearity}

\FunctionTok{raintest}\NormalTok{(tea\_model)}
\end{Highlighting}
\end{Shaded}

\begin{verbatim}
## 
##  Rainbow test
## 
## data:  tea_model
## Rain = 0.81268, df1 = 9, df2 = 7, p-value = 0.6229
\end{verbatim}

\begin{Shaded}
\begin{Highlighting}[]
\DocumentationTok{\#\#Linearity met}

\CommentTok{\#Assumption 4.3: Homoscedasticity}

\FunctionTok{bptest}\NormalTok{(tea\_model)}
\end{Highlighting}
\end{Shaded}

\begin{verbatim}
## 
##  studentized Breusch-Pagan test
## 
## data:  tea_model
## BP = 1.898, df = 1, p-value = 0.1683
\end{verbatim}

\begin{Shaded}
\begin{Highlighting}[]
\DocumentationTok{\#\#Homoscedasticity met}

\CommentTok{\#Assumption 4.4: Normality of residuals}

\FunctionTok{shapiro.test}\NormalTok{(tea\_model}\SpecialCharTok{$}\NormalTok{residuals)}
\end{Highlighting}
\end{Shaded}

\begin{verbatim}
## 
##  Shapiro-Wilk normality test
## 
## data:  tea_model$residuals
## W = 0.97873, p-value = 0.9359
\end{verbatim}

\begin{Shaded}
\begin{Highlighting}[]
\DocumentationTok{\#\#Normality of residuals met}

\CommentTok{\#All asumptions met}

\FunctionTok{glance}\NormalTok{(tea\_model)}
\end{Highlighting}
\end{Shaded}

\begin{verbatim}
## # A tibble: 1 x 12
##   r.squared adj.r.squared sigma statistic p.value    df logLik   AIC   BIC
##       <dbl>         <dbl> <dbl>     <dbl>   <dbl> <dbl>  <dbl> <dbl> <dbl>
## 1     0.303         0.260  1.22      6.96  0.0179     1  -28.0  62.0  64.7
## # i 3 more variables: deviance <dbl>, df.residual <int>, nobs <int>
\end{verbatim}

\begin{Shaded}
\begin{Highlighting}[]
\FunctionTok{summary}\NormalTok{(tea\_model)}
\end{Highlighting}
\end{Shaded}

\begin{verbatim}
## 
## Call:
## lm(formula = Tea ~ Coffee, data = counties_data_no_capital_log)
## 
## Residuals:
##     Min      1Q  Median      3Q     Max 
## -2.5048 -0.6359 -0.1003  0.9323  2.4137 
## 
## Coefficients:
##             Estimate Std. Error t value Pr(>|t|)  
## (Intercept)   4.0627     2.0916   1.942   0.0699 .
## Coffee        0.5756     0.2182   2.638   0.0179 *
## ---
## Signif. codes:  0 '***' 0.001 '**' 0.01 '*' 0.05 '.' 0.1 ' ' 1
## 
## Residual standard error: 1.217 on 16 degrees of freedom
##   (28 observations deleted due to missingness)
## Multiple R-squared:  0.3031, Adjusted R-squared:  0.2595 
## F-statistic: 6.958 on 1 and 16 DF,  p-value: 0.01791
\end{verbatim}

\begin{Shaded}
\begin{Highlighting}[]
\FunctionTok{library}\NormalTok{(apaTables)}
\NormalTok{apa\_table }\OtherTok{\textless{}{-}} \FunctionTok{apa.reg.table}\NormalTok{(tea\_model)}
\FunctionTok{print}\NormalTok{(apa\_table)}
\end{Highlighting}
\end{Shaded}

\begin{verbatim}
## 
## 
## Regression results using Tea as the criterion
##  
## 
##    Predictor     b      b_95%_CI beta  beta_95%_CI sr2 sr2_95%_CI    r
##  (Intercept)  4.06 [-0.37, 8.50]                                      
##       Coffee 0.58*  [0.11, 1.04] 0.55 [0.11, 0.99] .30 [.01, .56] .55*
##                                                                       
##                                                                       
##                                                                       
##              Fit
##                 
##                 
##       R2 = .303*
##  95% CI[.01,.56]
##                 
## 
## Note. A significant b-weight indicates the beta-weight and semi-partial correlation are also significant.
## b represents unstandardized regression weights. beta indicates the standardized regression weights. 
## sr2 represents the semi-partial correlation squared. r represents the zero-order correlation.
## Square brackets are used to enclose the lower and upper limits of a confidence interval.
## * indicates p < .05. ** indicates p < .01.
## 
\end{verbatim}

\begin{Shaded}
\begin{Highlighting}[]
\FunctionTok{library}\NormalTok{(rempsyc)}

\NormalTok{stats.table }\OtherTok{\textless{}{-}} \FunctionTok{as.data.frame}\NormalTok{(}\FunctionTok{summary}\NormalTok{(tea\_model)}\SpecialCharTok{$}\NormalTok{coefficients)}

\NormalTok{CI }\OtherTok{\textless{}{-}} \FunctionTok{confint}\NormalTok{(tea\_model)}

\NormalTok{stats.table }\OtherTok{\textless{}{-}} \FunctionTok{cbind}\NormalTok{(}\FunctionTok{row.names}\NormalTok{(stats.table), stats.table, CI)}

\FunctionTok{names}\NormalTok{(stats.table) }\OtherTok{\textless{}{-}} \FunctionTok{c}\NormalTok{(}\StringTok{"Variable"}\NormalTok{, }\StringTok{"Estimate"}\NormalTok{, }\StringTok{"Std. Error"}\NormalTok{, }\StringTok{"t{-}value"}\NormalTok{, }\StringTok{"p"}\NormalTok{, }\StringTok{"CI\_lower"}\NormalTok{, }\StringTok{"CI\_upper"}\NormalTok{)}

\FunctionTok{nice\_table}\NormalTok{(stats.table, }\AttributeTok{highlight =}\NormalTok{ .}\DecValTok{05}\NormalTok{)}
\end{Highlighting}
\end{Shaded}

\global\setlength{\Oldarrayrulewidth}{\arrayrulewidth}

\global\setlength{\Oldtabcolsep}{\tabcolsep}

\setlength{\tabcolsep}{0pt}

\renewcommand*{\arraystretch}{1.5}



\providecommand{\ascline}[3]{\noalign{\global\arrayrulewidth #1}\arrayrulecolor[HTML]{#2}\cline{#3}}

\begin{longtable}[c]{cccccc}



\hhline{>{\arrayrulecolor[HTML]{000000}\global\arrayrulewidth=0.5pt}->{\arrayrulecolor[HTML]{000000}\global\arrayrulewidth=0.5pt}->{\arrayrulecolor[HTML]{000000}\global\arrayrulewidth=0.5pt}->{\arrayrulecolor[HTML]{000000}\global\arrayrulewidth=0.5pt}->{\arrayrulecolor[HTML]{000000}\global\arrayrulewidth=0.5pt}->{\arrayrulecolor[HTML]{000000}\global\arrayrulewidth=0.5pt}-}

\multicolumn{1}{>{}c}{\textcolor[HTML]{000000}{\fontsize{12}{24}\selectfont{Variable}}} & \multicolumn{1}{>{}c}{\textcolor[HTML]{000000}{\fontsize{12}{24}\selectfont{Estimate}}} & \multicolumn{1}{>{}c}{\textcolor[HTML]{000000}{\fontsize{12}{24}\selectfont{Std.\ Error}}} & \multicolumn{1}{>{}c}{\textcolor[HTML]{000000}{\fontsize{12}{24}\selectfont{t-value}}} & \multicolumn{1}{>{}c}{\textcolor[HTML]{000000}{\fontsize{12}{24}\selectfont{\textit{p}}}} & \multicolumn{1}{>{}c}{\textcolor[HTML]{000000}{\fontsize{12}{24}\selectfont{95\%\ CI}}} \\

\noalign{\global\arrayrulewidth 0pt}\arrayrulecolor[HTML]{000000}

\hhline{>{\arrayrulecolor[HTML]{000000}\global\arrayrulewidth=0.5pt}->{\arrayrulecolor[HTML]{000000}\global\arrayrulewidth=0.5pt}->{\arrayrulecolor[HTML]{000000}\global\arrayrulewidth=0.5pt}->{\arrayrulecolor[HTML]{000000}\global\arrayrulewidth=0.5pt}->{\arrayrulecolor[HTML]{000000}\global\arrayrulewidth=0.5pt}->{\arrayrulecolor[HTML]{000000}\global\arrayrulewidth=0.5pt}-}\endhead



\multicolumn{1}{>{}l}{\textcolor[HTML]{000000}{\fontsize{12}{24}\selectfont{(Intercept)}}} & \multicolumn{1}{>{}c}{\textcolor[HTML]{000000}{\fontsize{12}{24}\selectfont{4.06}}} & \multicolumn{1}{>{}c}{\textcolor[HTML]{000000}{\fontsize{12}{24}\selectfont{2.09}}} & \multicolumn{1}{>{}c}{\textcolor[HTML]{000000}{\fontsize{12}{24}\selectfont{1.94}}} & \multicolumn{1}{>{}c}{\textcolor[HTML]{000000}{\fontsize{12}{24}\selectfont{.070}}} & \multicolumn{1}{>{}c}{\textcolor[HTML]{000000}{\fontsize{12}{24}\selectfont{[-0.37,\ 8.50]}}} \\

\noalign{\global\arrayrulewidth 0pt}\arrayrulecolor[HTML]{000000}





\multicolumn{1}{>{\cellcolor[HTML]{D9D9D9}}l}{\textcolor[HTML]{000000}{\fontsize{12}{24}\selectfont{\textbf{Coffee}}}} & \multicolumn{1}{>{\cellcolor[HTML]{D9D9D9}}c}{\textcolor[HTML]{000000}{\fontsize{12}{24}\selectfont{\textbf{0.58}}}} & \multicolumn{1}{>{\cellcolor[HTML]{D9D9D9}}c}{\textcolor[HTML]{000000}{\fontsize{12}{24}\selectfont{\textbf{0.22}}}} & \multicolumn{1}{>{\cellcolor[HTML]{D9D9D9}}c}{\textcolor[HTML]{000000}{\fontsize{12}{24}\selectfont{\textbf{2.64}}}} & \multicolumn{1}{>{\cellcolor[HTML]{D9D9D9}}c}{\textcolor[HTML]{000000}{\fontsize{12}{24}\selectfont{\textbf{.018*}}}} & \multicolumn{1}{>{\cellcolor[HTML]{D9D9D9}}c}{\textcolor[HTML]{000000}{\fontsize{12}{24}\selectfont{\textbf{[0.11,\ 1.04]}}}} \\

\noalign{\global\arrayrulewidth 0pt}\arrayrulecolor[HTML]{000000}

\hhline{>{\arrayrulecolor[HTML]{000000}\global\arrayrulewidth=0.5pt}->{\arrayrulecolor[HTML]{000000}\global\arrayrulewidth=0.5pt}->{\arrayrulecolor[HTML]{000000}\global\arrayrulewidth=0.5pt}->{\arrayrulecolor[HTML]{000000}\global\arrayrulewidth=0.5pt}->{\arrayrulecolor[HTML]{000000}\global\arrayrulewidth=0.5pt}->{\arrayrulecolor[HTML]{000000}\global\arrayrulewidth=0.5pt}-}



\end{longtable}



\arrayrulecolor[HTML]{000000}

\global\setlength{\arrayrulewidth}{\Oldarrayrulewidth}

\global\setlength{\tabcolsep}{\Oldtabcolsep}

\renewcommand*{\arraystretch}{1}

\begin{Shaded}
\begin{Highlighting}[]
\CommentTok{\#Issues: no R2, no asterisk next to p value, no Notes. section}
\end{Highlighting}
\end{Shaded}

\begin{Shaded}
\begin{Highlighting}[]
\CommentTok{\#H3: Farming is a mediator between population and average household size}

\FunctionTok{cor}\NormalTok{(counties\_data\_no\_capital}\SpecialCharTok{$}\NormalTok{Population, counties\_data\_no\_capital}\SpecialCharTok{$}\NormalTok{Farming, }\AttributeTok{use =} \StringTok{"pairwise.complete.obs"}\NormalTok{)}
\end{Highlighting}
\end{Shaded}

\begin{verbatim}
## [1] 0.7781674
\end{verbatim}

\begin{Shaded}
\begin{Highlighting}[]
\FunctionTok{cor}\NormalTok{(counties\_data\_no\_capital}\SpecialCharTok{$}\NormalTok{Population, counties\_data\_no\_capital}\SpecialCharTok{$}\NormalTok{AverageHouseholdSize, }\AttributeTok{use =} \StringTok{"pairwise.complete.obs"}\NormalTok{)}
\end{Highlighting}
\end{Shaded}

\begin{verbatim}
## [1] -0.1839505
\end{verbatim}

\begin{Shaded}
\begin{Highlighting}[]
\FunctionTok{cor}\NormalTok{(counties\_data\_no\_capital}\SpecialCharTok{$}\NormalTok{Farming, counties\_data\_no\_capital}\SpecialCharTok{$}\NormalTok{AverageHouseholdSize, }\AttributeTok{use =} \StringTok{"pairwise.complete.obs"}\NormalTok{)}
\end{Highlighting}
\end{Shaded}

\begin{verbatim}
## [1] -0.3254971
\end{verbatim}

\begin{Shaded}
\begin{Highlighting}[]
\NormalTok{mediate\_model }\OtherTok{\textless{}{-}} \FunctionTok{lm}\NormalTok{(Farming }\SpecialCharTok{\textasciitilde{}}\NormalTok{ Population, }\AttributeTok{data =}\NormalTok{ counties\_data\_no\_capital)}
\FunctionTok{summary}\NormalTok{(mediate\_model)}
\end{Highlighting}
\end{Shaded}

\begin{verbatim}
## 
## Call:
## lm(formula = Farming ~ Population, data = counties_data_no_capital)
## 
## Residuals:
##     Min      1Q  Median      3Q     Max 
## -157608  -22818    8139   34420   80570 
## 
## Coefficients:
##              Estimate Std. Error t value Pr(>|t|)    
## (Intercept) 1.981e+04  1.605e+04   1.235    0.224    
## Population  1.262e-01  1.535e-02   8.219 1.96e-10 ***
## ---
## Signif. codes:  0 '***' 0.001 '**' 0.01 '*' 0.05 '.' 0.1 ' ' 1
## 
## Residual standard error: 49200 on 44 degrees of freedom
## Multiple R-squared:  0.6055, Adjusted R-squared:  0.5966 
## F-statistic: 67.55 on 1 and 44 DF,  p-value: 1.962e-10
\end{verbatim}

\begin{Shaded}
\begin{Highlighting}[]
\NormalTok{full\_model }\OtherTok{\textless{}{-}} \FunctionTok{lm}\NormalTok{(AverageHouseholdSize }\SpecialCharTok{\textasciitilde{}}\NormalTok{ Population }\SpecialCharTok{+}\NormalTok{ Farming, }\AttributeTok{data =}\NormalTok{ counties\_data\_no\_capital)}
\FunctionTok{summary}\NormalTok{(full\_model)}
\end{Highlighting}
\end{Shaded}

\begin{verbatim}
## 
## Call:
## lm(formula = AverageHouseholdSize ~ Population + Farming, data = counties_data_no_capital)
## 
## Residuals:
##      Min       1Q   Median       3Q      Max 
## -1.89335 -0.51092  0.04188  0.48750  2.27275 
## 
## Coefficients:
##               Estimate Std. Error t value Pr(>|t|)    
## (Intercept)  4.677e+00  2.776e-01  16.847   <2e-16 ***
## Population   3.206e-07  4.158e-07   0.771   0.4450    
## Farming     -5.199e-06  2.564e-06  -2.027   0.0488 *  
## ---
## Signif. codes:  0 '***' 0.001 '**' 0.01 '*' 0.05 '.' 0.1 ' ' 1
## 
## Residual standard error: 0.8369 on 43 degrees of freedom
## Multiple R-squared:  0.1181, Adjusted R-squared:  0.07712 
## F-statistic:  2.88 on 2 and 43 DF,  p-value: 0.06701
\end{verbatim}

\begin{Shaded}
\begin{Highlighting}[]
\NormalTok{results }\OtherTok{\textless{}{-}} \FunctionTok{mediate}\NormalTok{(mediate\_model, full\_model, }\AttributeTok{treat =} \StringTok{"Population"}\NormalTok{, }\AttributeTok{mediator =} \StringTok{"Farming"}\NormalTok{, }\AttributeTok{robustSE =} \ConstantTok{TRUE}\NormalTok{, }\AttributeTok{sims =} \DecValTok{1000}\NormalTok{)}
\FunctionTok{summary}\NormalTok{(results)}
\end{Highlighting}
\end{Shaded}

\begin{verbatim}
## 
## Causal Mediation Analysis 
## 
## Quasi-Bayesian Confidence Intervals
## 
##                 Estimate 95% CI Lower 95% CI Upper p-value
## ACME           -6.52e-07    -1.93e-06          0.0    0.33
## ADE             3.32e-07    -1.36e-06          0.0    0.71
## Total Effect   -3.20e-07    -1.05e-06          0.0    0.39
## Prop. Mediated  8.35e-01    -2.23e+01         23.4    0.66
## 
## Sample Size Used: 46 
## 
## 
## Simulations: 1000
\end{verbatim}

\begin{Shaded}
\begin{Highlighting}[]
\FunctionTok{mediate\_plot}\NormalTok{(AverageHouseholdSize }\SpecialCharTok{\textasciitilde{}}\NormalTok{ Population }\SpecialCharTok{+}\NormalTok{ Farming, }\AttributeTok{data =}\NormalTok{ counties\_data\_no\_capital, }\AttributeTok{mediator =}\NormalTok{ Farming, }\AttributeTok{method =} \StringTok{"lm"}\NormalTok{)}
\end{Highlighting}
\end{Shaded}

\includegraphics{kenya_code_files/figure-latex/H3, Mediation analysis-1.pdf}

\end{document}
